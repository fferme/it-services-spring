%=====================================================================
 % Na introdução deve-se utilizar \chapter* e \section* 
\chapter*{1. Introdução}
%=====================================================================

\section*{1.1 Apresentação}

Este trabalho se propõe a facilitar a realização  do registro de clientes, ordens e ordens de serviço por parte do prestador de serviço responsável. A aplicação é construída seguindo o paradigma CRUD (Create, Read, Update, Delete), que permite a manipulação eficiente dos dados armazenados.

Para a implementação do projeto, foi utilizado o framework Java Spring Boot, que facilita o desenvolvimento de aplicações robustas e escaláveis. O versionamento de código é gerenciado pelo Git e GitHub, garantindo um controle eficiente das versões e a colaboração entre desenvolvedores.

O projeto utiliza Docker para a criação de ambientes de desenvolvimento e testes consistentes, e H2 como banco de dados de teste. Para o banco de dados de produção, foi adotado uma solução SQL integrada ao Docker. A qualidade do código é assegurada por meio de testes automatizados com JUnit, cobertura de testes com Jacoco e mutação de testes com Pitest. Maven é utilizado como gerenciador de dependências, facilitando a gestão e atualização das bibliotecas necessárias.

A aplicação adota a abordagem de serviços RESTful, garantindo uma comunicação eficiente e padronizada entre os componentes do sistema. A segurança é reforçada com a implementação de autenticação e autorização utilizando tokens JWT (JSON Web Token), que protegem os recursos da aplicação de acessos não autorizados. Além disso, o projeto incorpora o uso de um servidor de cache com Redis, melhorando significativamente o desempenho da aplicação.

A aplicação desenvolvida é altamente adaptável e customizável, podendo ser ajustada conforme as necessidades específicas de diferentes tipos de negócios. Isso a torna uma ferramenta versátil e escalável, adequada para diversas empresas no ramo da tecnologia, que buscam otimizar seus processos de registro e gerenciamento de clientes e serviços.
\\\\\\\\\\\\
\section*{1.2 Justificativa}

No atual cenário tecnológico, a gestão eficiente de informações é crucial para profissionais de TI, que lidam diariamente com uma vasta quantidade de dados, desde registros de clientes até ordens de serviço. A necessidade de um sistema robusto e confiável para gerenciar essas informações é imperativa, garantindo que todos os dados relevantes sejam armazenados de forma segura e acessível.

Profissionais de TI enfrentam desafios constantes relacionados à organização e manutenção de dados críticos. A falta de uma ferramenta adequada para gerenciar esses dados pode resultar em perda de informações, dificuldades no acesso e processamento de dados, além de problemas de segurança e conformidade. Um sistema bem estruturado é essencial para minimizar esses riscos e melhorar a eficiência operacional.

O próprio armazenamento de tais informações faz parte de um dos paradigmas do programador que é versionar todos os códigos. Ao conseguir manter e armazenar o histórico de ordens de serviço e clientes, o prestador de serviço é capaz de prestar um atendimento totalmente atencioso e personalizado (já que ele irá saber o que foi alterado no passado e possíveis preferências do cliente).\\\\\\\\\\\\\\\\\\\\\\\\\\\\\\\\\\\\\\

\section*{1.3 Objetivos}

\section*{1.3.1 Objetivo geral}

Desenvolver uma aplicação web robusta e eficiente para o registro e gerenciamento de clientes, ordens e ordens de serviço, utilizando a abordagem CRUD (Create, Read, Update, Delete) e adotando práticas de segurança e desempenho que atendam às necessidades de profissionais de TI.

\section*{1.3.1 Objetivos específicos}

Desenvolver uma API que permita a criação, leitura, atualização e exclusão de registros de clientes, ordens e ordens de serviço, garantindo a conformidade com os princípios REST para promover uma comunicação eficiente e padronizada entre os componentes do sistema.

Implementar autenticação e autorização utilizando tokens JWT para proteger os dados contra acessos não autorizados, adotando práticas recomendadas de segurança para assegurar a integridade e confidencialidade das informações.

Utilizar Git e GitHub para versionamento de código, promovendo a colaboração entre desenvolvedores e o controle de versões, e documentar a API de forma clara e acessível, utilizando ferramentas como Swagger, para facilitar a integração com outras aplicações.

Implementar um servidor de cache com Redis para melhorar o tempo de resposta da aplicação, e realizar testes automatizados com JUnit e cobertura de testes com Jacoco para garantir a eficiência e a estabilidade do código.

Realizar testes de mutação com Pitest para identificar e corrigir possíveis falhas no código, utilizando um banco de dados de teste H2 durante o desenvolvimento para simular condições reais e garantir a consistência dos dados.

Utilizar Docker para a criação de ambientes de desenvolvimento e teste consistentes, facilitando a escalabilidade da aplicação, e garantir que a aplicação seja adaptável e customizável para atender às necessidades específicas de diferentes tipos de negócios.

Desenvolver uma interface de usuário intuitiva e amigável que facilite a interação com a aplicação, proporcionando uma experiência eficiente e responsiva que atenda às expectativas dos profissionais de TI.

\section*{1.4 Estrutura do Trabalho}

















