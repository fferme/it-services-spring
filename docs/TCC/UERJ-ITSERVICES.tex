\documentclass[a4paper,12pt,oneside,onecolumn,final,fleqn]{repUERJ}
% ---
% Pacotes fundamentais 
% ---
\usepackage[english,brazil]{babel}  % adequação para o português Brasil
\usepackage[utf8]{inputenc} % Determina a codificação utilizada
                            % (conversão automática dos acentos)
\usepackage{makeidx}        % Cria o índice
\usepackage{hyperref}       % Controla a formação do índice
\usepackage{indentfirst}    % Indenta o primeiro paragrafo de
                            % cada seção.
\usepackage{graphicx}       % Inclusão de gráficos
\usepackage{subfig}
\usepackage{amsmath}        % pacote matemático
\usepackage{bm}             % pacote de fontes matemáticas
\usepackage{lscape}         % pacote para colocar páginas em modo paisagem
\usepackage{array}


% ---
% Pacote auxiliar para as normas da UERJ
% ---
\usepackage[frame=no,font=default]{repUERJformat}
\usepackage[line=yes]{repUERJpseudocode}
% ---
% Pacotes de citacoes
% ---
\usepackage[alf,abnt-repeated-author-omit=no]{abntex2cite}


\selectlanguage{brazil} % setando o idioma global para português



% ********************************************************************
% ********************************************************************
% Área Reservada para incluir os novos comandos
% ********************************************************************
% ********************************************************************

% Comandos de comentários
\definecolor{green}{rgb}{0.1, 0.4, 0.1}
\newcommand\red[1]{{\color{red}#1}}
\newcommand\sred[1]{{\color{red}\uline{#1}}}
\newcommand\blue[1]{{\color{blue}#1}}
\newcommand\sblue[1]{{\color{blue}\uline{#1}}}
\newcommand\green[1]{{\color{green}#1}}
\newcommand\sgreen[1]{{\color{green}\uline{#1}}}
\newcommand\orange[1]{{\color{orange}#1}}
\newcommand\sorange[1]{{\color{orange}\uline{#1}}}
%o comando \sout funciona para riscar o texto

% Abaixo tem-se alguns exemplos de definição de comandos

\newcommand\dd{\mathrm{d}} % d de derivada
\newcommand\DD{\mathrm{D}} % D de derivada
\newcommand\ee{\mathrm{e}} % número natural: e
\newcommand\ii{\mathrm{i}} % número complexo: i

\newcommand\Vol{\mathcal{V}} % Simbolo de volume

\newcommand\Vector[1]{\mbox{\boldmath$#1$}} % comando para colocar vetores
\newcommand\Tensor[1]{\mbox{\boldmath$\mathrm{#1}$}} % comando para colocar tensores/matrizes

%Alguns exemplos de definição de vetores, tensores e matrizes
\newcommand\vvec{\Vector{v}} % vetor
\newcommand\Tten{\Tensor{T}} % Tensor
\newcommand\Mmat{\Tensor{M}} % Matriz

% Alguns parâmetros adimensionais
\newcommand\Bi{\mathrm{Bi}} % Número de Biot
\newcommand\Rey{\mathrm{Re}} % Número de Reynolds
\newcommand\Ma{\mathrm{Ma}} % Número de Mach
\newcommand\Pe{\mathrm{Pe}} % Número de Péclet

\newcommand\Dh{D_H} % Diâmetro hidráulico

% Alguns índices máximos 
\newcommand\nmax{n_{\text{max}}}
\newcommand\mmax{m_{\text{max}}}
\newcommand\imax{i_{\text{max}}}
\newcommand\jmax{j_{\text{max}}}








% ********************************************************************
% ********************************************************************
% Informações de autoria e institucionais
% ********************************************************************
% ********************************************************************

%---------------------------------------------------------------------
% Imagens pretextuais (precisam estar no mesmo diretório deste arquivo .tex)
%---------------------------------------------------------------------

\logo{logo_uerj_cinza.png}
\marcadagua{marcadagua_uerj_cinza.png}{1}{160}{255}

%---------------------------------------------------------------------
% Informações da instituição
%---------------------------------------------------------------------
\instituicao{Universidade do Estado do Rio de Janeiro}  %Universidade
            {Centro de Tecnologia e Ciências}  %Centro
            {Faculdade de Ciências da Computação} %Unidade
            {Departamento de Computação} %Departamento

%---------------------------------------------------------------------
% Informações da autoria do documento
%---------------------------------------------------------------------

\autor{Felipe}
      {Ferme Cajueiro}
      {F. F. C.} % iniciais do nome

\titulo{O uso de Frameworks no desenvolvimento de Sistema para Gerenciamento de Clientes e Ordens de Serviços} %Título do trabalho acadêmico em português
\title{The use of Frameworks in the Development of a Customer and Service Order Management System} %Título do trabalho acadêmico em inglês





\orientador{Prof. Adriana Aparicio Sicsú Ayres do Nascimento} %cargo, ex.: Prof., Profa., Eng. , etc...
		   %{Daniel}{José Nahid Mansur Chalhub, DSc} %Nome sobrenome com a titulação ao final. Exemplo: D.Sc., Ph.D., M.Sc., B.Sc., etc.
           %{Universidade do Estado do Rio de Janeiro (UERJ) - DEPCOMP-COMP} % Instituição e Departamento ou PPG


%Opcional, Comente as linhas de coorientador caso não tenha
%\coorientador{cargo}  %cargo, ex.: Prof., Profa., Eng. , etc...
           %{nome}{sobrenome, titulação}  %Nome sobrenome com a titulação ao final. Exemplo: D.Sc., Ph.D., M.Sc., B.Sc., etc.
           %{unidade -- instituição} % Departamento ou PPG e Instituição 

%---------------------------------------------------------------------
% Grau pretendido (Doutor, Mestre, Bacharel, Licenciado) e Curso
%---------------------------------------------------------------------
%Descomente as sua opções

\newcommand\artigo{à}%Projeto de graduação
%\newcommand\artigo{ao}%Pós Graduação


\grau{Graduação}
%\grau{Mestre} 
%\grau{Doutor}


\curso{Ciências da Computação}
\newcommand\grauTitulo{Bacharel em Ciências da Computação}  


\areadeconcentracao{APIs} %Projeto de Graduação
%\areadeconcentracao{Fenômenos de Transporte} %Pós Graduação
%\areadeconcentracao{Mecânica dos Sólidos} %Pós Graduação



%---------------------------------------------------------------------
% Informações adicionais (local, data e paginas)
%---------------------------------------------------------------------

\local{Rio de Janeiro} 
\data{16}{Maio}{2024} % Exemplo: \data{21}{Março}{2016}

% ********************************************************************
% ********************************************************************
% Configurações de aparência do PDF final
% ********************************************************************
% ********************************************************************

% alterando o aspecto da cor azul
\definecolor{blue}{RGB}{41,5,195}
%\definecolor{apricot}{RGB}{251,206,177}

% informações do PDF
\hypersetup{
  unicode=false,
  pdftitle={\UERJtitulo},
  pdfauthor={\UERJautor},
  pdfsubject={\UERJpreambulo},
  pdfkeywords={PALAVRAS}{CHAVES}{\chaveA}{\chaveB}{\chaveC}{\chaveD},
  pdfproducer={\packagename}, % producer of the document
  pdfcreator={\UERJautor},
  colorlinks=true,            % false: boxed links; true: colored links
  linkcolor=black,            % color of internal links blue
  citecolor=black,            % color of links to bibliography blue
  filecolor=black,            % color of file links magenta
  urlcolor=black,
  bookmarksdepth=4,
  %backref=true,
  %pagebackref=true,
  %bookmarks=true,
}

% ********************************************************************
% ********************************************************************
% Início do documento
% ********************************************************************
% ********************************************************************
% ---
% compila o índice; se não for usar, comentar
% ---
\makeindex
% ********************************************************************
% ********************************************************************
\begin{document}
% ----------------------------------------------------------
%% ELEMENTOS PRE-TEXTUAIS
% ----------------------------------------------------------
\frontmatter
% ----------------------------------------------------------
% Capa e a folha de rosto
% ----------------------------------------------------------
\capa
\folhaderosto
% ----------------------------------------------------------
% Inserir a ficha catalográfica
% ----------------------------------------------------------
% ---
% A biblioteca deverá providenciar a ficha catalográfica. Salve a ficha no
% formato PDF. Use o nome do arquivo PDF como argumento do comando. 
% Exemplo: ficha catalográfica é o arquivo 'Ficha.pdf' na pasta "B.PreTextual"
%     \fichacatalografica{B.PreTextual/Ficha.pdf}
%
% Enquanto não possuir a ficha catalográfica, use o comando sem argumentos.
% ---
\fichacatalografica{B.PreTextual/Ficha.pdf}



% ----------------------------------------------------------
% Folha de aprovação
% ----------------------------------------------------------

%Após obter a assinatura dos membros da banca, comente as linhas a baixos e insira o pdf com as assinaturas na pasta "B.PreTextual"


% Ajustar o espaço entre as assinaturas abaixo
\newcommand{\spc}{0.25cm}

\begin{folhadeaprovacao}
	\vspace{\spc} % Espaço entre assinaturas
	\assinatura{Professor numero 2}{Universidade do Estado do Rio de Janeiro (UERJ)} %Exemplo
	\vspace{\spc} % Espaço entre assinaturas
	\assinatura{Professor numero 3}{Universidade do Estado do Rio de Janeiro (UERJ)}
	\vspace{\spc} % Espaço entre assinaturas
	\assinatura{Professor numero 4}{Universidade do Estado do Rio de Janeiro (UERJ)}
	\vspace{\spc} % Espaço entre assinaturas
	\assinatura{Professor numero 5}{Universidade do Estado do Rio de Janeiro (UERJ)}
\end{folhadeaprovacao}


% Após colocar o pdf com as assinaturas na pasta "B.PreTextual", comente todo o ambiente "folhadeaprovacao" acima, descomente a linha abaixo e insira o nome correto do arquivo pdf:
%\includepdf[pages=1]{B.PreTextual/ficha.pdf} %exemplo


% ----------------------------------------------------------
% Dedicatória
\pretextualchapter{Dedicatória}
\vfill
Dedico este TCC aos meus pais, que nunca deixaram faltar instrumentos de estudo ou de trabalho durante minha trajetória. De mesmo modo, aos meus amigos que sempre se puseram disponíveis para tirar minhas dúvidas em eventuais projetos.




% ----------------------------------------------------------
% Agradecimentos
\pretextualchapter{Agradecimentos}
Agradeço aos amigos que fiz na faculdade que sempre estiveram ao meu lado, oferecendo seu apoio incondicional e amizade sincera. Em momentos difíceis, eles foram minha âncora, lembrando-me do valor da persistência e da força que reside em mim mesmo.



% ----------------------------------------------------------
% Epigrafe (opcional)
\pretextualchapter{}
\vfill
\begin{flushright}
	A criatividade é a inteligência se divertindo\\
	\textit{Albert Einstein}
\end{flushright}










% ----------------------------------------------------------
%% RESUMO
% se não for usar a quarta palavra chave, deixar o campo vazio: {}
\palavraschaves
{TCC Java Spring Boot}
{API RESTful}
{Java}
{Spring Framework}



\pretextualchapter{Resumo}
\referencia % linha em branco depois

Este trabalho se propõe a realizar o registro de clientes, ordens e ordens de serviço por parte do prestador de serviço responsável. A aplicação é construída seguindo o paradigma CRUD (Create, Read, Update, Delete).

O projeto utiliza o framework Java Spring Boot para implementação, Git e GitHub para versionamento de código, Docker e H2 como bancos de dados de teste (SQL para Docker), testes automatizados com JUnit, Jacoco e Pitest, Maven como gerenciador de dependências. Além diso, adota a abordagem de serviços RESTFUL, realiza autenticação e autorização com Token JWT.

A aplicação desenvolvida pode ser adaptada e customizada de acordo com as necessidades específicas de diferentes tipos de negócio, representando uma ferramenta versátil e escalável para diversas empresas no ramo da tecnologia.\\

\imprimirchaves % linha em branco antes




% ----------------------------------------------------------
% Abstract
\begin{otherlanguage}{english}
  \keywords{first keyword}
{second keyword}
{third keyword}
{fourth keyword (if any)}


\pretextualchapter{Abstract}
\reference % linha em branco depois

% O resumo em inglês deve ser organizado em apenas um parágrafo mesmo.

This project aims to register clients, orders, and service orders by the responsible service provider. The application is built following the CRUD (Create, Read, Update, Delete) paradigm.

The project utilizes the Java Spring Boot framework for implementation, Git and GitHub for code versioning, Docker and H2 as test databases (SQL for Docker), automated testing with JUnit, Jacoco, and Pitest, Maven as a dependency manager. Additionally, it adopts the RESTFUL services approach, performs authentication and authorization with JWT Token. It also utilizes Redis server for cached data.

The developed application can be adapted and customized according to the specific needs of different types of businesses, representing a versatile and scalable tool for various technology companies.\\

\printkeys % linha em branco antes


\end{otherlanguage}



% ----------------------------------------------------------
% Listas de ilustrações e tabelas
% ----------------------------------------------------------
%\listadefiguras
%\listadetabelas
% ----------------------------------------------------------
% Outras listas
% ----------------------------------------------------------
%\listadealgoritmos


% ----------------------------------------------------------
% Lista de abreviaturas e siglas
%\include{B.PreTextual/06.AbbreviationList}

% ----------------------------------------------------------
% Lista de simbolos
%\include{B.PreTextual/07.SymbolList}

% ----------------------------------------------------------
% Sumario
\sumario

% ----------------------------------------------------------
%% ELEMENTOS TEXTUAIS
% ----------------------------------------------------------
\mainmatter



\renewcommand{\chaptername}{ASDASD}
%=====================================================================
 % Na introdução deve-se utilizar \chapter* e \section* 
\chapter*{1. Introdução}
%=====================================================================

\section*{1.1 Apresentação}

Este trabalho se propõe a facilitar a realização  do registro de clientes, ordens e ordens de serviço por parte do prestador de serviço responsável. A aplicação é construída seguindo o paradigma CRUD (Create, Read, Update, Delete), que permite a manipulação eficiente dos dados armazenados.

Para a implementação do projeto, foi utilizado o framework Java Spring Boot, que facilita o desenvolvimento de aplicações robustas e escaláveis. O versionamento de código é gerenciado pelo Git e GitHub, garantindo um controle eficiente das versões e a colaboração entre desenvolvedores.

O projeto utiliza Docker para a criação de ambientes de desenvolvimento e testes consistentes, e H2 como banco de dados de teste. Para o banco de dados de produção, foi adotado uma solução SQL integrada ao Docker. A qualidade do código é assegurada por meio de testes automatizados com JUnit, cobertura de testes com Jacoco e mutação de testes com Pitest. Maven é utilizado como gerenciador de dependências, facilitando a gestão e atualização das bibliotecas necessárias.

A aplicação adota a abordagem de serviços RESTful, garantindo uma comunicação eficiente e padronizada entre os componentes do sistema. A segurança é reforçada com a implementação de autenticação e autorização utilizando tokens JWT (JSON Web Token), que protegem os recursos da aplicação de acessos não autorizados. Além disso, o projeto incorpora o uso de um servidor de cache com Redis, melhorando significativamente o desempenho da aplicação.

A aplicação desenvolvida é altamente adaptável e customizável, podendo ser ajustada conforme as necessidades específicas de diferentes tipos de negócios. Isso a torna uma ferramenta versátil e escalável, adequada para diversas empresas no ramo da tecnologia, que buscam otimizar seus processos de registro e gerenciamento de clientes e serviços.
\\\\\\\\\\\\
\section*{1.2 Justificativa}

No atual cenário tecnológico, a gestão eficiente de informações é crucial para profissionais de TI, que lidam diariamente com uma vasta quantidade de dados, desde registros de clientes até ordens de serviço. A necessidade de um sistema robusto e confiável para gerenciar essas informações é imperativa, garantindo que todos os dados relevantes sejam armazenados de forma segura e acessível.

Profissionais de TI enfrentam desafios constantes relacionados à organização e manutenção de dados críticos. A falta de uma ferramenta adequada para gerenciar esses dados pode resultar em perda de informações, dificuldades no acesso e processamento de dados, além de problemas de segurança e conformidade. Um sistema bem estruturado é essencial para minimizar esses riscos e melhorar a eficiência operacional.

O próprio armazenamento de tais informações faz parte de um dos paradigmas do programador que é versionar todos os códigos. Ao conseguir manter e armazenar o histórico de ordens de serviço e clientes, o prestador de serviço é capaz de prestar um atendimento totalmente atencioso e personalizado (já que ele irá saber o que foi alterado no passado e possíveis preferências do cliente).\\\\\\\\\\\\\\\\\\\\\\\\\\\\\\\\\\\\\\

\section*{1.3 Objetivos}

\section*{1.3.1 Objetivo geral}

Desenvolver uma aplicação web robusta e eficiente para o registro e gerenciamento de clientes, ordens e ordens de serviço, utilizando a abordagem CRUD (Create, Read, Update, Delete) e adotando práticas de segurança e desempenho que atendam às necessidades de profissionais de TI.

\section*{1.3.1 Objetivos específicos}

Desenvolver uma API que permita a criação, leitura, atualização e exclusão de registros de clientes, ordens e ordens de serviço, garantindo a conformidade com os princípios REST para promover uma comunicação eficiente e padronizada entre os componentes do sistema.

Implementar autenticação e autorização utilizando tokens JWT para proteger os dados contra acessos não autorizados, adotando práticas recomendadas de segurança para assegurar a integridade e confidencialidade das informações.

Utilizar Git e GitHub para versionamento de código, promovendo a colaboração entre desenvolvedores e o controle de versões, e documentar a API de forma clara e acessível, utilizando ferramentas como Swagger, para facilitar a integração com outras aplicações.

Implementar um servidor de cache com Redis para melhorar o tempo de resposta da aplicação, e realizar testes automatizados com JUnit e cobertura de testes com Jacoco para garantir a eficiência e a estabilidade do código.

Realizar testes de mutação com Pitest para identificar e corrigir possíveis falhas no código, utilizando um banco de dados de teste H2 durante o desenvolvimento para simular condições reais e garantir a consistência dos dados.

Utilizar Docker para a criação de ambientes de desenvolvimento e teste consistentes, facilitando a escalabilidade da aplicação, e garantir que a aplicação seja adaptável e customizável para atender às necessidades específicas de diferentes tipos de negócios.

Desenvolver uma interface de usuário intuitiva e amigável que facilite a interação com a aplicação, proporcionando uma experiência eficiente e responsiva que atenda às expectativas dos profissionais de TI.

\section*{1.4 Estrutura do Trabalho}


















\chapter{Formulação Matemática}

\section{Fundamentação}

Lorem ipsum dolor sit amet, consectetur adipiscing elit, sed do eiusmod tempor incididunt ut labore et dolore magna aliqua. Ut enim ad minim veniam, quis nostrud exercitation ullamco laboris nisi ut aliquip ex ea commodo consequat. Duis aute irure dolor in reprehenderit in voluptate velit esse cillum dolore eu fugiat nulla pariatur. Excepteur sint occaecat cupidatat non proident, sunt in culpa qui officia deserunt mollit anim id est laborum.

Lorem ipsum dolor sit amet, consectetur adipiscing elit, sed do eiusmod tempor incididunt ut labore et dolore magna aliqua. Ut enim ad minim veniam, quis nostrud exercitation ullamco laboris nisi ut aliquip ex ea commodo consequat. Duis aute irure dolor in reprehenderit in voluptate velit esse cillum dolore eu fugiat nulla pariatur. Excepteur sint occaecat cupidatat non proident, sunt in culpa qui officia deserunt mollit anim id est laborum.

Lorem ipsum dolor sit amet, consectetur adipiscing elit, sed do eiusmod tempor incididunt ut labore et dolore magna aliqua. Ut enim ad minim veniam, quis nostrud exercitation ullamco laboris nisi ut aliquip ex ea commodo consequat. Duis aute irure dolor in reprehenderit in voluptate velit esse cillum dolore eu fugiat nulla pariatur. Excepteur sint occaecat cupidatat non proident, sunt in culpa qui officia deserunt mollit anim id est laborum.

\section{Metodologia}

As equações \eqref{eq:taylor-v1}, \eqref{eq:taylor-v2} e \eqref{eq:taylor-v2}.....
 
lorem ipsum dolor sit amet, consectetur adipiscing elit, sed do eiusmod tempor incididunt ut labore et dolore magna aliqua. Ut enim ad minim veniam, quis nostrud exercitation ullamco laboris nisi ut aliquip ex ea commodo consequat. Duis aute irure dolor in reprehenderit in voluptate velit esse cillum dolore eu fugiat nulla pariatur. Excepteur sint occaecat cupidatat non proident, sunt in culpa qui officia deserunt mollit anim id est laborum.
%
\begin{subequations}
\begin{gather}
v_r(t) = 0 \label{eq:taylor-v1} \\
v_{\theta}(t) = \varpi r \exp \left( - \dfrac{r^2}{4r_cRe^{-1}}t \right)  \label{eq:taylor-v2} \\
v_z(t) = 0 \label{eq:taylor-v3}
\end{gather}
\end{subequations}

As equações \eqref{eq:taylor-vx}, \eqref{eq:taylor-vy} e \eqref{eq:taylor-vz}...

lorem ipsum dolor sit amet, consectetur adipiscing elit, sed do eiusmod tempor incididunt ut labore et dolore magna aliqua. Ut enim ad minim veniam, quis nostrud exercitation ullamco laboris nisi ut aliquip ex ea commodo consequat. Duis aute irure dolor in reprehenderit in voluptate velit esse cillum dolore eu fugiat nulla pariatur. Excepteur sint occaecat cupidatat non proident, sunt in culpa qui officia deserunt mollit anim id est laborum.
%
\begin{subequations}
\begin{gather}
v_x(t) =  U_{ref} - v_{\theta}(t)\sin(\theta) \label{eq:taylor-vx} \\
v_y(t) =  v_{\theta}(t)\cos(\theta) \label{eq:taylor-vy} \\
v_z(t) = 0  \label{eq:taylor-vz}
\end{gather}
\end{subequations}

De acordo com a \ref{tbl:taylor-vortex-parameters} pode-se concluir...
 
Lorem ipsum dolor sit amet, consectetur adipiscing elit, sed do eiusmod tempor incididunt ut labore et dolore magna aliqua. Ut enim ad minim veniam, quis nostrud exercitation ullamco laboris nisi ut aliquip ex ea commodo consequat. Duis aute irure dolor in reprehenderit in voluptate velit esse cillum dolore eu fugiat nulla pariatur. Excepteur sint occaecat cupidatat non proident, sunt in culpa qui officia deserunt mollit anim id est laborum.
%
\begin{table}[h]{6cm}
\caption{Parâmetros Físicos}
\label{tbl:taylor-vortex-parameters}
\begin{tabular}{ll}
\hline
Parameter & Value \\
\hline
$ r_c $          	 & $ L_{ref}/30 $ \\
$ U_{ref} $          & $ 1 $	      \\
$ \varpi $           & $ 1 $		  \\
$ Re $               & $ 35 $  		  \\
$ Sc $               & $ 650 $  	  \\
$ \Delta t $         & $ 0.1 $        \\
\hline
\end{tabular}
%\legend{Texto da legenda. (Opcional)}
\source{Citação da fonte ou `O autor'. (opcional)}
\end{table}

Lorem ipsum dolor sit amet, consectetur adipiscing elit, sed do eiusmod tempor incididunt ut labore et dolore magna aliqua. Ut enim ad minim veniam, quis nostrud exercitation ullamco laboris nisi ut aliquip ex ea commodo consequat. Duis aute irure dolor in reprehenderit in voluptate velit esse cillum dolore eu fugiat nulla pariatur. Excepteur sint occaecat cupidatat non proident, sunt in culpa qui officia deserunt mollit anim id est laborum.
%
\begin{equation}
{\phi}|_{\Gamma_L} = {\phi}|_{\Gamma_R}
\end{equation}

Lorem ipsum dolor sit amet, consectetur adipiscing elit, sed do eiusmod tempor incididunt ut labore et dolore magna aliqua. Ut enim ad minim veniam, quis nostrud exercitation ullamco laboris nisi ut aliquip ex ea commodo consequat. Duis aute irure dolor in reprehenderit in voluptate velit esse cillum dolore eu fugiat nulla pariatur. Excepteur sint occaecat cupidatat non proident, sunt in culpa qui officia deserunt mollit anim id est laborum.



\chapter{Resultados}


Sed ut perspiciatis unde omnis iste natus error sit voluptatem accusantium doloremque laudantium, totam rem aperiam, eaque ipsa quae ab illo inventore veritatis et quasi architecto beatae vitae dicta sunt explicabo. Nemo enim ipsam voluptatem quia voluptas sit aspernatur aut odit aut fugit, sed quia consequuntur magni dolores eos qui ratione voluptatem sequi nesciunt. Neque porro quisquam est, qui dolorem ipsum quia dolor sit amet, consectetur, adipisci velit, sed quia non numquam eius modi tempora incidunt ut labore et dolore magnam aliquam quaerat voluptatem.

Sed ut perspiciatis unde omnis iste natus error sit voluptatem accusantium doloremque laudantium, totam rem aperiam, eaque ipsa quae ab illo inventore veritatis et quasi architecto beatae vitae dicta sunt explicabo. Nemo enim ipsam voluptatem quia voluptas sit aspernatur aut odit aut fugit, sed quia consequuntur magni dolores eos qui ratione voluptatem sequi nesciunt. Neque porro quisquam est, qui dolorem ipsum quia dolor sit amet, consectetur, adipisci velit, sed quia non numquam eius modi tempora incidunt ut labore et dolore magnam aliquam quaerat voluptatem.

Sed ut perspiciatis unde omnis iste natus error sit voluptatem accusantium doloremque laudantium, totam rem aperiam, eaque ipsa quae ab illo inventore veritatis et quasi architecto beatae vitae dicta sunt explicabo. Nemo enim ipsam voluptatem quia voluptas sit aspernatur aut odit aut fugit, sed quia consequuntur magni dolores eos qui ratione voluptatem sequi nesciunt. Neque porro quisquam est, qui dolorem ipsum quia dolor sit amet, consectetur, adipisci velit, sed quia non numquam eius modi tempora incidunt ut labore et dolore magnam aliquam quaerat voluptatem.

Sed ut perspiciatis unde omnis iste natus error sit voluptatem accusantium doloremque laudantium, totam rem aperiam, eaque ipsa quae ab illo inventore veritatis et quasi architecto beatae vitae dicta sunt explicabo. Nemo enim ipsam voluptatem quia voluptas sit aspernatur aut odit aut fugit, sed quia consequuntur magni dolores eos qui ratione voluptatem sequi nesciunt. Neque porro quisquam est, qui dolorem ipsum quia dolor sit amet, consectetur, adipisci velit, sed quia non numquam eius modi tempora incidunt ut labore et dolore magnam aliquam quaerat voluptatem.

Sed ut perspiciatis unde omnis iste natus error sit voluptatem accusantium doloremque laudantium, totam rem aperiam, eaque ipsa quae ab illo inventore veritatis et quasi architecto beatae vitae dicta sunt explicabo. Nemo enim ipsam voluptatem quia voluptas sit aspernatur aut odit aut fugit, sed quia consequuntur magni dolores eos qui ratione voluptatem sequi nesciunt. Neque porro quisquam est, qui dolorem ipsum quia dolor sit amet, consectetur, adipisci velit, sed quia non numquam eius modi tempora incidunt ut labore et dolore magnam aliquam quaerat voluptatem.

Sed ut perspiciatis unde omnis iste natus error sit voluptatem accusantium doloremque laudantium, totam rem aperiam, eaque ipsa quae ab illo inventore veritatis et quasi architecto beatae vitae dicta sunt explicabo. Nemo enim ipsam voluptatem quia voluptas sit aspernatur aut odit aut fugit, sed quia consequuntur magni dolores eos qui ratione voluptatem sequi nesciunt. Neque porro quisquam est, qui dolorem ipsum quia dolor sit amet, consectetur, adipisci velit, sed quia non numquam eius modi tempora incidunt ut labore et dolore magnam aliquam quaerat voluptatem.


A figura \ref{fig:LinhasDeCorrente} mostra as linhas de corrente...



\begin{figure}[!h]{0.5\textwidth}
	\caption{Linhas de corrente} \label{fig:LinhasDeCorrente}
	\includegraphics[width=\hsize]{Figures/StreamLines.pdf}
	%\legend{Texto da legenda. (opcional)}
	\source{Citação da fonte ou `O autor' (opcional)}
\end{figure}



Sed ut perspiciatis unde omnis iste natus error sit voluptatem accusantium doloremque laudantium, totam rem aperiam, eaque ipsa quae ab illo inventore veritatis et quasi architecto beatae vitae dicta sunt explicabo. Nemo enim ipsam voluptatem quia voluptas sit aspernatur aut odit aut fugit, sed quia consequuntur magni dolores eos qui ratione voluptatem sequi nesciunt. Neque porro quisquam est, qui dolorem ipsum quia dolor sit amet, consectetur, adipisci velit, sed quia non numquam eius modi tempora incidunt ut labore et dolore magnam aliquam quaerat voluptatem.



Texto da primeira seção. A Figura \ref{rotulo} é o logo da UERJ. 

\begin{figure}[!h]{5cm}
	\caption{Legenda da figura.} \label{rotulo}
	\includegraphics[width=\hsize]{Figures/logo_uerj_cor.jpg}
	%\legend{Texto da legenda. (opcional)}
	\source{Citação da fonte ou `O autor'. (opcional)}
\end{figure}

Sed ut perspiciatis unde omnis iste natus error sit voluptatem accusantium doloremque laudantium, totam rem aperiam, eaque ipsa quae ab illo inventore veritatis et quasi architecto beatae vitae dicta sunt explicabo. Nemo enim ipsam voluptatem quia voluptas sit aspernatur aut odit aut fugit, sed quia consequuntur magni dolores eos qui ratione voluptatem sequi nesciunt. Neque porro quisquam est, qui dolorem ipsum quia dolor sit amet, consectetur, adipisci velit, sed quia non numquam eius modi tempora incidunt ut labore et dolore magnam aliquam quaerat voluptatem.

Sed ut perspiciatis unde omnis iste natus error sit voluptatem accusantium doloremque laudantium, totam rem aperiam, eaque ipsa quae ab illo inventore veritatis et quasi architecto beatae vitae dicta sunt explicabo. Nemo enim ipsam voluptatem quia voluptas sit aspernatur aut odit aut fugit, sed quia consequuntur magni dolores eos qui ratione voluptatem sequi nesciunt. Neque porro quisquam est, qui dolorem ipsum quia dolor sit amet, consectetur, adipisci velit, sed quia non numquam eius modi tempora incidunt ut labore et dolore magnam aliquam quaerat voluptatem.


Texto da primeira subseção. Figura \ref{outro.rotulo}\subref{subrotulo2}.


\begin{figure}[!h]{8cm}
    \centering
	\caption{Curvas de Nusselt} \label{outro.rotulo}
	\subfloat[][Legenda a...]{\label{subrotulo1}
		\fbox{\includegraphics[width=0.45\hsize]{Figures/logo_uerj_cinza.png}}}
	\subfloat[][Legenda b...]{\label{subrotulo2}
		\fbox{\includegraphics[width=0.45\hsize]{Figures/marcadagua_uerj_cinza.png}}}\\
	\subfloat[][Legenda c...]{\label{subrotulo3}
		\fbox{\includegraphics[width=0.45\hsize]{Figures/logo_uerj_cor.jpg}}}
	\subfloat[][Legenda d...]{\label{subrotulo4}
		\fbox{\includegraphics[width=0.45\hsize]{Figures/logo_uerj_cinza.png}}}
	%\legend{Texto da legenda (opcional)}
	\source{Citação da fonte ou `O autor'. (opcional)}
\end{figure}


Sed ut perspiciatis unde omnis iste natus error sit voluptatem accusantium doloremque laudantium, totam rem aperiam, eaque ipsa quae ab illo inventore veritatis et quasi architecto beatae vitae dicta sunt explicabo. Nemo enim ipsam voluptatem quia voluptas sit aspernatur aut odit aut fugit, sed quia consequuntur magni dolores eos qui ratione voluptatem sequi nesciunt. Neque porro quisquam est, qui dolorem ipsum quia dolor sit amet, consectetur, adipisci velit, sed quia non numquam eius modi tempora incidunt ut labore et dolore magnam aliquam quaerat voluptatem.


Sed ut perspiciatis unde omnis iste natus error sit voluptatem accusantium doloremque laudantium, totam rem aperiam, eaque ipsa quae ab illo inventore veritatis et quasi architecto beatae vitae dicta sunt explicabo. Nemo enim ipsam voluptatem quia voluptas sit aspernatur aut odit aut fugit, sed quia consequuntur magni dolores eos qui ratione voluptatem sequi nesciunt. Neque porro quisquam est, qui dolorem ipsum quia dolor sit amet, consectetur, adipisci velit, sed quia non numquam eius modi tempora incidunt ut labore et dolore magnam aliquam quaerat voluptatem.


Texto da primeira subsubseção. Tabela \ref{mais.rotulo}.

\begin{table}[!h]{6cm}
	\caption{Título da tabela.}\label{mais.rotulo}
	\centering
	\renewcommand\arraystretch{1.0}
	\begin{tabular}{l|l}
		\hline
		X & Y\\
		\hline
		1,20 & 15,7\\
		1,23 & 15,6\\
		1,19 & 15,3\\
		1,26 & 15,1\\
		1,22 & 15,5\\
		1,16 & 15,3\\
		1,37 & 15,7\\
		\hline
	\end{tabular}
	%\legend{Texto da legenda. (opcional)}
	\source{Citação da fonte ou `O autor'. (opcional)}
\end{table}

Sed ut perspiciatis unde omnis iste natus error sit voluptatem accusantium doloremque laudantium, totam rem aperiam, eaque ipsa quae ab illo inventore veritatis et quasi architecto beatae vitae dicta sunt explicabo. Nemo enim ipsam voluptatem quia voluptas sit aspernatur aut odit aut fugit, sed quia consequuntur magni dolores eos qui ratione voluptatem sequi nesciunt. Neque porro quisquam est, qui dolorem ipsum quia dolor sit amet, consectetur, adipisci velit, sed quia non numquam eius modi tempora incidunt ut labore et dolore magnam aliquam quaerat voluptatem.


Sed ut perspiciatis unde omnis iste natus error sit voluptatem accusantium doloremque laudantium, totam rem aperiam, eaque ipsa quae ab illo inventore veritatis et quasi architecto beatae vitae dicta sunt explicabo. Nemo enim ipsam voluptatem quia voluptas sit aspernatur aut odit aut fugit, sed quia consequuntur magni dolores eos qui ratione voluptatem sequi nesciunt. Neque porro quisquam est, qui dolorem ipsum quia dolor sit amet, consectetur, adipisci velit, sed quia non numquam eius modi tempora incidunt ut labore et dolore magnam aliquam quaerat voluptatem.








\begin{table}[h!]{16cm}
	\caption{Legenda da tabela... \label{tab1}}
	\centering
	%\scriptsize\vspace{-0.5em}
	\renewcommand\arraystretch{1.0}
	\begin{tabular}{|c | c|c|c|c|c|c|c|}
		\hline
		$\xi$ & $\Pe=1$ & $\Pe=2$ & $\Pe=5$ & $\Pe=10$ & $\Pe=20$ & $\Pe=50$ & $\Pe=10^6$  \\
		\hline
		0.01 & 5.22296 & 5.20378 & 5.76517 & 6.51798 & 7.14765 & 7.25370 & 5.97331 \\
		0.02 & 5.22312 & 5.17788 & 5.55262 & 5.94688 & 6.07004 & 5.68785 & 5.01764 \\
		0.03 & 5.22859 & 5.15789 & 5.39565 & 5.56665 & 5.44551 & 5.00668 & 4.63233 \\
		0.04 & 5.23425 & 5.14160 & 5.26962 & 5.28439 & 5.04655 & 4.65836 & 4.43743 \\
		0.05 & 5.24000 & 5.12761 & 5.16310 & 5.06644 & 4.78235 & 4.46566 & 4.33066 \\
		0.1 & 5.26629 & 5.06903 & 4.79158 & 4.50409 & 4.29634 & 4.21132 & 4.19648 \\
		0.2 & 5.28890 & 4.94549 & 4.46973 & 4.26465 & 4.20200 & 4.18899 & 4.18681 \\
		0.3 & 5.27099 & 4.83267 & 4.37396 & 4.23631 & 4.19915 & 4.18883 & 4.18676 \\
		0.4 & 5.22959 & 4.75309 & 4.34476 & 4.23275 & 4.19907 & 4.18883 & 4.18676 \\
		0.5 & 5.18075 & 4.70280 & 4.33558 & 4.23229 & 4.19907 & 4.18883 & 4.18676 \\
		1 & 5.01217 & 4.63313 & 4.33127 & 4.23223 & 4.19907 & 4.18883 & 4.18676 \\
		2 & 4.94867 & 4.62757 & 4.33126 & 4.23223 & 4.19907 & 4.18883 & 4.18676 \\
		5 & 4.94445 & 4.62753 & 4.33126 & 4.23223 & 4.19907 & 4.18883 & 4.18676 \\
		10 & 4.94445 & 4.62753 & 4.33126 & 4.23223 & 4.19907 & 4.18883 & 4.18676 \\
		20 & 4.94445 & 4.62753 & 4.33126 & 4.23223 & 4.19907 & 4.18883 & 4.18676 \\
		50 & 4.94445 & 4.62753 & 4.33126 & 4.23223 & 4.19907 & 4.18883 & 4.18676 \\
		100 & 4.94445 & 4.62753 & 4.33126 & 4.23223 & 4.19907 & 4.18883 & 4.18676 \\
		\hline 
	\end{tabular}
\end{table}



Sed ut perspiciatis unde omnis iste natus error sit voluptatem accusantium doloremque laudantium, totam rem aperiam, eaque ipsa quae ab illo inventore veritatis et quasi architecto beatae vitae dicta sunt explicabo. Nemo enim ipsam voluptatem quia voluptas sit aspernatur aut odit aut fugit, sed quia consequuntur magni dolores eos qui ratione voluptatem sequi nesciunt. Neque porro quisquam est, qui dolorem ipsum quia dolor sit amet, consectetur, adipisci velit, sed quia non numquam eius modi tempora incidunt ut labore et dolore magnam aliquam quaerat voluptatem.



Sed ut perspiciatis unde omnis iste natus error sit voluptatem accusantium doloremque laudantium, totam rem aperiam, eaque ipsa quae ab illo inventore veritatis et quasi architecto beatae vitae dicta sunt explicabo. Nemo enim ipsam voluptatem quia voluptas sit aspernatur aut odit aut fugit, sed quia consequuntur magni dolores eos qui ratione voluptatem sequi nesciunt. Neque porro quisquam est, qui dolorem ipsum quia dolor sit amet, consectetur, adipisci velit, sed quia non numquam eius modi tempora incidunt ut labore et dolore magnam aliquam quaerat voluptatem.

\begin{landscape}
	\begin{figure}[!h]{20cm}
    \centering
	\caption{Exemplo de Página em landscape} \label{outro.rotulo2}
    %\hspace*{-6cm}
	\subfloat[][Legenda a...]{\label{subrotulo5}
		\fbox{\includegraphics[width=0.5\hsize]{Figures/Plot2.pdf}}}
	\subfloat[][Legenda b...]{\label{subrotulo6}
		\fbox{\includegraphics[width=0.5\hsize]{Figures/Plot3.pdf}}}
    \\ %\hspace*{-6cm}
	\subfloat[][Legenda c...]{\label{subrotulo7}
		\fbox{\includegraphics[width=0.5\hsize]{Figures/Plot4.pdf}}}
	\subfloat[][Legenda d...]{\label{subrotulo8}
		\fbox{\includegraphics[width=0.5\hsize]{Figures/Plot1.pdf}}}
	%\legend{Texto da legenda (opcional)}
	\source{Citação da fonte ou `O autor'. (opcional)}
	\end{figure}
\end{landscape}








\include{C.Chapters/99.Conclusion}




% ----------------------------------------------------------
%% ELEMENTOS POS-TEXTUAIS
% ----------------------------------------------------------
\backmatter
%=====================================================================
% Referencias via BibTeX
%=====================================================================
\citeoption{abnt-options4}
\bibliography{A.Bibliography/MyBibliography}
%=====================================================================




%=====================================================================
%% Glossário
%=====================================================================
%=====================================================================
\postextualchapter*{Glossário}
%=====================================================================


\definicao{termo 1}{significado}
\definicao{termo 2}{significado}
\definicao{termo 3}{significado}




% ----------------------------------------------------------
% Apêndices (opcionais)
% ----------------------------------------------------------
% ---
% Inicia os apêndices
% ---
\appendix

%=====================================================================
\postextualchapter{Primeiro apêndice}
%=====================================================================
\section{Primeira seção}

Texto da primeira seção\index{Introdução!Capítulo}.

\subsection{Primeira subseção}

Texto da primeira subseção.

\subsubsection{Primeira subsubseção}

Texto da primeira subsubseção.

\include{D.PostTextual/02.Appendix-B}






% ----------------------------------------------------------
% Anexos (opcionais)
% ----------------------------------------------------------
% ---
% Inicia os anexos
% ---
\annex

%=====================================================================
\postextualchapter{Primeiro anexo}
%=====================================================================

Modelo de trabalho acadêmico utilizando classe repUERJ para elaboração de teses, dissertação e monografias em geral (projetos finais e trabalhos de conclusão de curso).

Este modelo foi criado por Dr. Luís Fernando de Oliveira, Professor Adjunto do Departamento de Física Aplicada e Termodinâmica, Instituto de Física Armando Dias Tavares da Universidade do Estado do Rio de Janeiro -- UERJ

A classe repUERJ.cls foi criada a partir do código original disponibilizado pelo grupo CódigoLivre.Org (equipe coordenada por Gerald Weber). Foram feitas adequações para implementação das normas de elaboração de teses e dissertações da UERJ.

Os estilos repUERJformat.sty codificam os elementos pré-textuais e pós-textuais.

O estilo repUERJpseudocode.sty codifica a elaboração de algoritmos utilizando um glossário desenvolvido por mim (Luís Fernando), o mesmo usado em meu curso de Física Computacional.

Este arquivo está editado na codificação de caracteres UTF-8.

As referencia estão baseadas no modelo bibtex e citação em autor-data.

Todo este material está disponível também no meu site \url{http://sites.google.com/site/deoliveiralf}.

As normas da UERJ para elaboração de teses e dissertações pode ser obtidas no documento disponível no site \url{http://www.bdtd.uerj.br/roteiro_uerj_web.pdf}.

Agradecimentos ao NPROTEC/Rede Sirius/UERJ e à Biblioteca Setorial da Física.


\include{D.PostTextual/03.Annex-B}



%---------------------------------------------------------------------
%% INDICE REMISSIVO (relativo ao makeindex)
%---------------------------------------------------------------------
\printindex
%=====================================================================
\end{document}
