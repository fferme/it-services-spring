%=====================================================================
\postextualchapter{Primeiro anexo}
%=====================================================================

Modelo de trabalho acadêmico utilizando classe repUERJ para elaboração de teses, dissertação e monografias em geral (projetos finais e trabalhos de conclusão de curso).

Este modelo foi criado por Dr. Luís Fernando de Oliveira, Professor Adjunto do Departamento de Física Aplicada e Termodinâmica, Instituto de Física Armando Dias Tavares da Universidade do Estado do Rio de Janeiro -- UERJ

A classe repUERJ.cls foi criada a partir do código original disponibilizado pelo grupo CódigoLivre.Org (equipe coordenada por Gerald Weber). Foram feitas adequações para implementação das normas de elaboração de teses e dissertações da UERJ.

Os estilos repUERJformat.sty codificam os elementos pré-textuais e pós-textuais.

O estilo repUERJpseudocode.sty codifica a elaboração de algoritmos utilizando um glossário desenvolvido por mim (Luís Fernando), o mesmo usado em meu curso de Física Computacional.

Este arquivo está editado na codificação de caracteres UTF-8.

As referencia estão baseadas no modelo bibtex e citação em autor-data.

Todo este material está disponível também no meu site \url{http://sites.google.com/site/deoliveiralf}.

As normas da UERJ para elaboração de teses e dissertações pode ser obtidas no documento disponível no site \url{http://www.bdtd.uerj.br/roteiro_uerj_web.pdf}.

Agradecimentos ao NPROTEC/Rede Sirius/UERJ e à Biblioteca Setorial da Física.
