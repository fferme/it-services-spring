% se não for usar a quarta palavra chave, deixar o campo vazio: {}
\palavraschaves
{TCC Java Spring Boot}
{API RESTful}
{Java}
{Spring Framework}



\pretextualchapter{Resumo}
\referencia % linha em branco depois

Este trabalho se propõe a realizar o registro de clientes, ordens e ordens de serviço por parte do prestador de serviço responsável. A aplicação é construída seguindo o paradigma CRUD (Create, Read, Update, Delete).

O projeto utiliza o framework Java Spring Boot para implementação, Git e GitHub para versionamento de código, Docker e H2 como bancos de dados de teste (SQL para Docker), testes automatizados com JUnit, Jacoco e Pitest, Maven como gerenciador de dependências. Além diso, adota a abordagem de serviços RESTFUL, realiza autenticação e autorização com Token JWT. Também utiliza um servidor em cache com Redis.

A aplicação desenvolvida pode ser adaptada e customizada de acordo com as necessidades específicas de diferentes tipos de negócio, representando uma ferramenta versátil e escalável para diversas empresas no ramo da tecnologia.\\

\imprimirchaves % linha em branco antes


